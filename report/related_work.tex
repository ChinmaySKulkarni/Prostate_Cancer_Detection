\section{Related Work}
There have been many different approaches tried for cancer detection using automated techniques. In \cite{automatic}, authors process MRI images of prostate region for detecting cancer. They make use of Haralick texture features \cite{haralick1973textural} in order to extract texture-based features and then use SVM for recursive feature selection which they apply to the entire image and evaluate their results using 10-fold cross-validation. The approach does not extract any features from the image other than the Haralick features or make use of any information about the structure of various components of the images. 

The authors of \cite{naik2007gland} make use of a more comprehensive approach. They segment parts of the image into detected gland components and subsequently extract morphological features from these components. To reduce the dimensionality of their data, they employ graph embedding and manifold learning. The features extracted for each image are very generic and do not capture much information specific to the differences in shape and size of various components of the image.

In \cite{tabesh2007multifeature}, the authors make use of both morphological features as well as histogram-based features. They pre-process the images so that background colors are removed and so that the staining is made uniform. The striking fact about their approach is that they do not make use of any morphological features present in the various components of the image. In spite of using histogram-based features for their classification, the authors remove white pixels corresponding to lumen which is not intuitive. 

\cite{alexandratou2010evaluation} make use of Haralick features for extracting 13 texture characteristics using the Grey Level Co-occurence Matrix for images. They experiment with 16 different classifiers and run these classifiers over a very small data set or only about 40 images for cancer and non-cancer tissues. They obtain accuracy values of around 95\% for some of their classifiers, but the data set they use for classification is too small for this accuracy to make much sense. The authors also count accuracy values for differentiating Gleason Score 3 and Gleason Score 4 images when they report their overall accuracy, when in fact this kind of distinction is a much easier task than differentiating cancerous images from non-cancerous images.

\cite{roula2002multispectral} make use of a novel approach wherein instead of analyzing RGB or grey scale images, they make use of the same feature vectors for a set of 16 spectral color bands. The authors apply a supervised classical Linear Discrimination method on the PCA result of the resultant combined feature vector (which is a linear combination of feature vectors for each color band) and show that the classification accuracy improves when using this spectral approach. The original feature vector contained texture-based features extracted using Haralick features. They observe better performance using the combined feature vector for 16 spectral bands over the original feature vector. The main drawback of this approach is that they are tweaking the data collection itself by observing each image as a combined feature vector of its 16 different spectral bands.

In most of the work done so far, not much effort has been put into capturing morphological features that are representative of different components of the image (lumen glands, epithelial cells and stroma). Most approaches discussed above make use of generic texture-based features without any specific processing or feature engineering for different image segments. Moreover, except for the last approach, all approaches make use of a limited number of images for testing their classifiers. In our work, we aim to focus the bulk of our effort on constructing effective and intuitive features for solving this problem. This leads to more robust features and our approach thus yields high accuracy values over data sets which are larger than those used in previous work.
\label{sec:rel_work}