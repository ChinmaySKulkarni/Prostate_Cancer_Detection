\section{Conclusions}
\label{sec:conclusions}

In this work, we examined the problem of detecting prostrate cancer using stained tissue images. Conversations with pathologists revealed the presence of certain signs that are highly indicative of cancer in many (but not all) cases. This domain knowledge was used in engineering relevant features. These features were able to improve the classifier accuracy by XX\%. On a large dataset, our techniques showed promising accuracy of 82\%. However, cancer detection still remains a challenging problem. While this work has shown that incorporating domain knowledge helps, further work is needed to identify other tissue components like basal cells, white blood cells. Additionally, these images are recorded at varying degrees of maginification -- pathologists often zoom in to see nucleus-level features when other evidence is inconclusive. Thus, incorporating multi-resolution features into the modelling also merits further exploration. 